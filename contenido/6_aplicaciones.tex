\section{Aplicaciones potenciales}

% Describe cómo puede usarse tu base de datos:
% - Modelado forestal
% - Clasificación de usos del suelo
% - Análisis temporal de cambios
% - Entrenamiento de algoritmos de machine learning
% - Estudios de conservación, captura de carbono, etc.

%ESTO ESTA DE CHATI DIRECTAMENTE, NO ME GUSTA DEMASIADO. 

La base de datos desarrollada ofrece un conjunto de variables ambientales, estructurales y espaciales altamente integradas, lo que la convierte en una herramienta versátil para múltiples disciplinas y objetivos de análisis dentro del ámbito forestal, ecológico y ambiental. A continuación se describen algunas de las aplicaciones potenciales más destacadas:

\begin{itemize}
    \item \textbf{Modelización forestal y crecimiento}: la estructura jerárquica por parcela, especie y clase diamétrica permite construir modelos de crecimiento individual o agregado, facilitando la simulación de escenarios forestales bajo distintos regímenes de manejo o condiciones ambientales.

    \item \textbf{Clasificación de usos del suelo}: la combinación de información estructural y espectral, junto con variables climáticas estacionales, permite entrenar modelos de clasificación para diferenciar entre tipos de cobertura y usos del suelo.

    \item \textbf{Análisis temporal de cambios}: la disponibilidad de registros para múltiples inventarios nacionales (IFN2, IFN3 e IFN4) posibilita el seguimiento de dinámicas forestales a lo largo del tiempo, como cambios en la composición de especies, incremento en la biomasa o variaciones estructurales a nivel de parcela.

    \item \textbf{Entrenamiento de algoritmos de aprendizaje automático}: el conjunto rico y estructurado de variables (climáticas, edáficas, espectrales y dendrométricas) es apto para alimentar modelos de \textit{Machine Learning} en tareas de predicción, clasificación o segmentación espacial.

    \item \textbf{Evaluación de servicios ecosistémicos}: mediante los indicadores de carbono (aéreo, subterráneo y total), la base de datos puede contribuir a estimaciones de captura de carbono, almacenamiento de biomasa y su evolución a escala regional o nacional.

    \item \textbf{Estudios de conservación y restauración}: al contener variables de origen, ocupación, estado y tratamientos de masas forestales, permite identificar áreas de conservación prioritaria, evaluar la efectividad de tratamientos selvícolas o detectar masas degradadas susceptibles de restauración.

    \item \textbf{Desarrollo de indicadores territoriales}: la estructura georreferenciada y la resolución a nivel de parcela favorecen la agregación de indicadores a distintas escalas (municipal, provincial o regional), contribuyendo a la planificación forestal o a la formulación de políticas públicas basadas en evidencia.
\end{itemize}

La interoperabilidad con otras fuentes de datos geoespaciales y la posibilidad de actualización periódica aseguran que esta base de datos pueda mantenerse como una infraestructura científica viva, abierta a nuevas aplicaciones y necesidades de análisis futuras.


