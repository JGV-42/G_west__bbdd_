\section{Anexos}
\small


\subsection{Anexo: Códigos de provincias de España}\label{sec:provincias}

\begin{table}[H]
\centering
\renewcommand{\arraystretch}{1.3}
\begin{tabular}{|c|l||c|l|}
\hline
\textbf{Código} & \textbf{Provincia} & \textbf{Código} & \textbf{Provincia} \\
\hline
01 & Álava             & 27 & Lugo \\
02 & Albacete          & 28 & Madrid \\
03 & Alicante          & 29 & Málaga \\
04 & Almería           & 30 & Murcia \\
05 & Ávila             & 31 & Navarra \\
06 & Badajoz           & 32 & Ourense \\
07 & Baleares (Illes)  & 33 & Asturias \\
08 & Barcelona         & 34 & Palencia \\
09 & Burgos            & 35 & Palmas (Las) \\
10 & Cáceres           & 36 & Pontevedra \\
11 & Cádiz             & 37 & Salamanca \\
12 & Castellón         & 38 & Santa Cruz de Tenerife \\
13 & Ciudad Real       & 39 & Cantabria \\
14 & Córdoba           & 40 & Segovia \\
15 & Coruña (A)        & 41 & Sevilla \\
16 & Cuenca            & 42 & Soria \\
17 & Girona            & 43 & Tarragona \\
18 & Granada           & 44 & Teruel \\
19 & Guadalajara       & 45 & Toledo \\
20 & Guipúzcoa         & 46 & Valencia \\
21 & Huelva            & 47 & Valladolid \\
22 & Huesca            & 48 & Vizcaya \\
23 & Jaén              & 49 & Zamora \\
24 & León              & 50 & Zaragoza \\
25 & Lleida            & 51 & Ceuta \\
26 & La Rioja          & 52 & Melilla \\
\hline
\end{tabular}
\caption{Relación de códigos numéricos utilizados para las provincias españolas.}
\label{anexo:provincias}
\end{table}


\subsection{Anexo: Rocosidad}\label{sec:Rocosid}
Se considerará el conjunto de la parcela clasificando la rocosidad según la siguiente codificación:

\begin{enumerate}
    \item \textbf{Sin pedregosidad}: la superficie de la parcela está completamente cubierta de vegetación.
    \item \textbf{Poco pedregoso}: cuando la superficie de la parcela cubierta por rocas coherentes es menor del 25\,\%.
    \item \textbf{Pedregoso}: cuando la superficie rocosa está comprendida entre el 25\,\% y el 50\,\%.
    \item \textbf{Muy pedregoso}: cuando la superficie rocosa se sitúa entre el 50\,\% y el 75\,\%.
    \item \textbf{Roquedo}: cuando la superficie de rocas es mayor del 75\,\%. En este caso, no se tomará ningún dato adicional correspondiente a suelos.
\end{enumerate}





\subsection{Anexo: Tipo de Suelo}\label{sec:TipSuelo}

Se utilizará la siguiente codificación para el tipo de suelo, diferenciando tres variables:

\vspace{1em}
\noindent
\textbf{Tipo de suelo (I):} \textbf{Presencia de sales, yesos o hidromorfía}

\begin{enumerate}
    \item \textbf{No se observan sales, yesos ni procesos de fidromorfía.}
    \item \textbf{Suelo salino.} Si presenta al menos dos de las siguientes características:
    \begin{itemize}
        \item Presencia de eflorescencias en la superficie o a distintas profundidades.
        \item Existencia de plantas halófitas.
        \item Zonas llanas o endorreicas con climas secos que provocan gran evaporación.
    \end{itemize}
    
    
    \item \textbf{Suelo yesífero.} Si presenta alguna de las siguientes características:
    \begin{itemize}
        \item Presencia de materia yesífera en superficie o a distintas profundidades.
        \item Existencia de plantas gipsófilas.
    \end{itemize}
    
    
    \item \textbf{Suelo hidromorfo.} Si el suelo presenta síntomas de hidromorfía acusada, cumpliendo al menos dos de las siguientes:
    \begin{itemize}
        \item Zona encharcada permanente o casi permanentemente de forma natural.
        \item Zona llana o endorreica con climas húmedos.
        \item Grietas en verano si no hay encharcamiento.
        \item Presencia de vegetación indicadora de hidromorfismo.
    \end{itemize}
\end{enumerate}

Identificandose las siguientes:
\begin{itemize}
    \item Formaciones vegetales indicadoras de hidromorfía:
    \begin{itemize}
        \item Ribereñas: \textit{saucedas, mimbreras, alisedas}.
        \item Brezales con \textit{Erica ciliaris, Erica tetralix}.
        \item Turberas arboladas (excepto Cornisa Cantábrica y Pirineos).
        \item Turberas de montaña con \textit{Sphagnum, Erica tetralix}.
        \item Cervunales con \textit{Nardus stricta}.
        \item Carrizales y espadañares (\textit{Phragmites, Tipha, Cladium}).
        \item Juncales (\textit{Scirpus, Juncus}).
        \item Pastizales con cárices (\textit{Carex spp.}).
        \item Marismas.
    \end{itemize}
    \item Formaciones vegetales gipsófilas:
    \begin{itemize}
        \item Aznallar: matorral de \textit{Ononis tridentata}.
        \item Tomillares gipsófilos con:
        \begin{itemize}
            \item \textit{Lepidium subulatum}
            \item \textit{Gypsophila spp.}
            \item \textit{Matthiola fruticulosa}
        \end{itemize}
    \end{itemize}
    \item   Formaciones vegetales indicadoras de suelos salinos:
    \begin{itemize}
        \item Salicorniales: matas leñosas crasas (Salicornia, Arthrocnemum, Halozylon).
        \item Bosques halófitos del género \textit{Tamarix}.
        \item Saladar o sosar: predominio de \textit{Suaeda vera}.
        \item Saladar blanco: predominio de \textit{Atriplex halimus}.
    \end{itemize}
\end{itemize}

    
\vspace{1em}
\noindent
\textbf{Tipo de suelo (II y III):} \textbf{Composición del suelo (calizo o silíceo)}

\begin{enumerate}
    \item \textbf{Suelo calizo.} Más del 50\,\% de la vertical del perfil da efervescencia con ácido clorhídrico.
    
    \begin{itemize}
        \item \textbf{Moderadamente básico:} pH en superficie $\leq 8.5$.
        \item \textbf{Fuertemente básico:} pH en superficie $> 8.5$.
    \end{itemize}
    
    \item \textbf{Suelo silíceo.} Menos del 50\,\% de la vertical del perfil da efervescencia.
    
    \begin{itemize}
        \item \textbf{Moderadamente ácido:} pH $\geq 5.5$.
        \item \textbf{Fuertemente ácido:} pH $< 5.5$.
    \end{itemize}
\end{enumerate}

\subsection{Anexo: Manifestaciones Erosivas}\label{sec:ManERo}

Se observará la parcela y sus alrededores hasta una distancia de 60 metros desde el centro, y se codificará la existencia de manifestaciones erosivas según la siguiente clave:

\begin{enumerate}
    \item \textbf{No hay ninguna manifestación.}
    
    \item \textbf{Cuellos de raíces al descubierto:} los cuellos de las raíces están visibles, con acumulación de residuos aguas arriba de los tallos y obstáculos, así como abundancia superficial de piedras.
    
    \item \textbf{Presencia de regueros:} canales paralelos de erosión con una profundidad máxima de un palmo (aproximadamente 20 cm).
    
    \item \textbf{Cárcavas y barrancos en V:} erosión lineal más profunda que los regueros, con forma de ``V''.
    
    \item \textbf{Cárcavas y barrancos en U:} erosión avanzada con formas suavizadas y amplias en ``U''.
    
    \item \textbf{Deslizamientos del terreno:} desplazamientos de masas de tierra, ladera o materiales del suelo.
\end{enumerate}

\subsection{Anexo: Distribución Espacial}\label{sec:disEsp}

La disposición de la vegetación en el espacio se clasificará según la siguiente codificación:

\begin{enumerate}
    \item \textbf{Uniforme.} Cuando el estrato arbóreo presenta continuidad en el espacio.

    \item \textbf{Diseminada en bosquetes aislados.} Cuando la masa arbórea se encuentra dividida en porciones que tienen una superficie inferior a 0,5 ha.

    \item \textbf{Diseminada en individuos aislados.} Cuando se trata de dehesas.

    \item[9.] \textbf{Otras o no se sabe.} En caso diferente a los anteriores o si se desconoce el dato exacto.
\end{enumerate}


\subsection{Anexo: Composición Específica}\label{anexo:compesp}

En función de las especies presentes:

\begin{enumerate}
    \item \textbf{Masas homogéneas o puras}. Masas monoespecíficas con una única especie arbórea. La normativa española precisa que una masa es monoespecífica o pura cuando al menos el 90\% de los pies pertenecen a la misma especie.
    
    \item \textbf{Masas heterogéneas o mezcladas pie a pie}. Masas de diferentes especies que se juntan o bien se entremezclan por golpes o grupos, siempre que tengan una altura similar.
    
    \item \textbf{Masas heterogéneas o mezcladas con subpiso}. Las dos o más especies mezcladas, cuando alcancen el estado adulto y la estabilidad, presentarán alturas diferentes.
    
    \item[9.] \textbf{Otras o no se sabe}. En caso diferente a los anteriores o desconocer el dato exacto.
\end{enumerate}



\subsection{Anexo: Textura del Suelo}\label{sec:textura}

Se clasificará en función de la siguiente codificación:

\begin{enumerate}
    \item \textbf{Suelo arenoso.} Si los cilindros se deshacen sin apenas formarse.
    \item \textbf{Suelo franco.} Es posible hacer cilindros gruesos pero no delgados.
    \item \textbf{Suelo arcilloso.} Se consiguen cilindros de unos 5 mm de diámetro.
\end{enumerate}

\subsection{Anexo: Nivel de usos del suelo}\label{sec:nivel1}

\begin{enumerate}
    \item \textbf{Monte.} Toda superficie en la que vegetan especies arbóreas, arbustivas, de matorral o herbáceas, ya sea espontáneamente o procedan de siembra o plantación, siempre que no sean características de cultivo agrícola o fueran objeto del mismo.
    \item \textbf{Agrícola.} Territorio o ecosistema poblado con siembras o plantaciones de herbáceas y/o leñosas, anuales o plurianuales que se laborea con una fuerte intervención humana, puede estar poblado por especies forestales de fruto (flor, hojas o en el futuro biomasa) siempre que la intervención humana sea importante. Incluye las dehesas, montes huecos o montes adehesados de base cultivo, siempre que la fracción de cabida cubierta de los árboles sea inferior al 5\%.
    \item \textbf{Artificial.} Territorio o ecosistemas dominado por edificios, parques urbanos (aunque estén poblados de árboles), viveros fuera de los montes (aunque sean de especies forestales), carreteras (salvo las vías de servicio de los montes) u otras construcciones humanas que tengan superficies continuas.
    \item \textbf{Humedal.} Lo constituyen las lagunas, charcas, zonas húmedas, marismas y corrientes discontinuas de agua en las que, al menos durante 6 meses del año, esté presente dicho líquido.
    \item \textbf{Agua.} Es la parte de la tierra constituida por ríos, lagos, embalses, canales o estanques con superficies continuas de más de 0.26 ha y con agua prácticamente todo el año.
\end{enumerate}


\subsection{Anexo: Nivel morfoestructural}\label{sec:nivel2}
Para el nivel de usos del suelo Monte se definirán los siguientes niveles morfoestructurales.

\begin{enumerate}
    \item \textbf{Monte arbolado.} Territorio o ecosistema con especies forestales arbóreas como manifestación vegetal de estructura vertical dominante y con una fracción de cabida cubierta igual o superior al 20\%; incluye dehesas con base cultivo o pastizal con labores siempre que la fracción arbolada supere el 20\%, y excluye terrenos con fuerte intervención humana para obtener frutos, hojas, flores o varas.
    
    \item \textbf{Monte arbolado ralo.} Terreno de uso forestal con especies arbóreas forestales dominantes y fracción de cabida cubierta entre el 10\% y 20\% (incluido el 10\%, excluido el 20\%); también aplica a terrenos con matorral o pastizal natural como dominantes, pero con presencia importante de árboles forestales, incluyendo dehesas de base de cultivo.
    
    \item \textbf{Monte temporalmente desarbolado.} Terreno que fue monte arbolado recientemente y que casi con seguridad volverá a estar cubierto de árboles en un futuro próximo.
    
    \item \textbf{Monte desarbolado.} Terreno con matorral y/o pastizal natural o débil intervención humana como cobertura dominante, con fracción de cabida cubierta por árboles forestales inferior al 5\%.
    
    \item \textbf{Monte sin vegetación superior.} Terreno de uso forestal que no está poblado por vegetales superiores debido a condiciones actuales de suelo, clima o topografía, aunque podría estarlo en otras circunstancias.
    
    \item \textbf{Árboles fuera del monte.} Incluye riberas arboladas no estructuradas con los montes, bosquetes de menos de 2.500 m\textsuperscript{2}, alineaciones de especies arbóreas o arbustivas de menos de 25 m de anchura, y árboles sueltos en terreno forestal.
    
    \item \textbf{Monte arbolado disperso.} Terreno forestal con especies arbóreas dominantes y fracción de cabida cubierta entre el 5\% y el 10\% (incluido el 5\%, excluido el 10\%); también terrenos con matorral o pastizal como cobertura dominante pero con presencia significativa de árboles forestales, incluyendo dehesas de base cultivo.
\end{enumerate}

\subsection{Anexo: Contenido en Materia Orgánica}\label{sec:MatOrg}


Según la siguiente clasificación:

\begin{enumerate}
    \item \textbf{Suelo muy humífero.} Cuando a 15 cm la pureza es menor de 4, o cuando la capa de broza sea de espesor mayor de 5 cm y a 15 cm de profundidad la pureza sea menor de 6.
    \item \textbf{Suelo moderadamente humífero.} Cuando a 15 cm la pureza sea menor de 6 con capa de broza nula o de escaso espesor, o cuando dicha capa tenga espesor mayor de 5 cm y a 15 cm de profundidad la pureza sea igual o mayor de 6.
    \item \textbf{Suelo poco humífero.} En los restantes casos.
\end{enumerate}



\subsection{Anexo: Reacción del Suelo (pH)}\label{sec:ph}

En función del pH, el suelo se clasifica según la siguiente codificación:

\begin{table}[H]
\centering
\renewcommand{\arraystretch}{1.4}
\begin{tabular}{|c|l|c|}
\hline
\textbf{Valores del pH de la solución del suelo} & \textbf{Clasificación del suelo} & \textbf{Codificación} \\
\hline
$1$     & Suelo extremadamente ácido       & $1$ \\
$2$     & Suelo muy fuertemente ácido      & $2$ \\
$3-4$   & Suelo fuertemente ácido          & $3$ \\
$5-6$   & Suelo moderadamente ácido        & $4$ \\
$7 $    & Suelo neutro                     & $5$ \\
$8 $    & Suelo moderadamente básico       & $6$ \\
$9 $    & Suelo fuertemente básico         & $7$ \\
$10 $   & Suelo extremadamente básico      & $8$ \\
\hline
\end{tabular}
\caption{Clasificación del suelo según el pH de la solución del suelo.}
\label{anexo:ph}
\end{table}


\subsection{Anexo: Espesor de la Capa Muerta, Césped, Musgo y Líquenes}\label{sec:EspMue}

Se anotará con la siguiente codificación:

\begin{itemize}
    \item Espesor menor de $0,5$ cm \hfill \textbf{00}
    \item Espesor de $0,5$ a $1,4$ cm \hfill \textbf{01}
    \item Espesor de $1,5$ a $2,4$ cm \hfill \textbf{02}
    \item Espesor de $2,5$ a $3,4$ cm \hfill \textbf{03}
    \item \textit{Y así sucesivamente.}
\end{itemize}

\vspace{1em}
\noindent
Si en la parcela hay zonas con diferentes espesores de capa muerta, se anotará el valor medio estimado.

\subsection{Anexo: Espesor de la Capa Muerta, Césped, Musgo y Líquenes}\label{sec:modComb}
Se determinará la clase de combustible que es más probable que propague el fuego si hubiese un incendio en la zona, hasta un máximo de 60m: pasto, matorral, hojarasca de bosque o deshechos o restos de corta. Se determinará el modelo de combustible a partir de la siguiente clave:

\begin{table}[H]
\renewcommand{\arraystretch}{2.2}
\centering
\resizebox{\textwidth}{!}{%
\begin{tabular}{|>{\centering\arraybackslash}m{8cm}|>{\centering\arraybackslash}m{2cm}|m{14cm}|}
\hline
\cellcolor[HTML]{D9EAD3}{\color[HTML]{000000}\textbf{GRUPO}} &
\cellcolor[HTML]{D9EAD3}{\color[HTML]{000000}\textbf{MOD. COMBUSTIBLE}} &
\cellcolor[HTML]{D9EAD3}{\color[HTML]{000000}\textbf{DESCRIPCIÓN DEL MODELO}} \\
\hline

\multirow{3}{*}{PASTOS} 
& 1 & Pasto fino, seco y bajo, que recubre completamente el suelo. Puede aparecer algunas plantas leñosas dispersas ocupando menos de 1/3 de la superficie. \\
\cline{2-3}
& 2 & Pasto fino, seco y bajo, que recubre completamente el suelo. Las plantas leñosas dispersas cubren de 1/3 a 2/3 de la superficie; pero la propagación del fuego se realiza por el pasto. \\
\cline{2-3}
& 3 & Pasto grueso, denso, seco y alto (> 1 m). Puede haber algunas plantas leñosas dispersas. Los campos de cereales son representativos de este modelo. \\
\hline

\multirow{4}{*}{MATORRAL} 
& 4 & Matorral o plantación joven muy densa; de más de 2 m de altura; con ramas muertas en su interior. Propagación del fuego por las copas de las plantas. \\
\cline{2-3}
& 5 & Matorral disperso, denso y verde, de menos de 1 m de altura. Propagación del fuego por la hojarasca, el pasto, las ramillas y el matorral. \\
\cline{2-3}
& 6 & Parecido al modelo 5, pero con especies más inflamables, de mayor talla, pudiéndose encontrar ramas gruesas en el suelo. Propagación del fuego con vientos moderados a fuertes. \\
\cline{2-3}
& 7 & Matorral de especies muy inflamables; de 0.5 a 2 m de altura, situado como sotobosque en masas de coníferas. \\
\hline

\multirow{3}{*}{HOJARASCA\\BAJO ARBOLADO} 
& 8 & Bosque denso, sin matorral. Propagación del fuego por la hojarasca muy compacta, formada por acículas cortas (5 cm o menos) o por hojas planas no muy grandes. \\
\cline{2-3}
& 9 & Parecido al modelo 8, pero con hojarasca menos compacta, formada por acículas largas y rígidas (P. pinaster) o follaje de frondosas de hoja grande, caducas (castaño o robles). \\
\cline{2-3}
& 10 & Bosque con gran cantidad de leña y árboles caídos, como consecuencia de vendavales, plagas intensas, etc. \\
\hline

\multirow{3}{*}{RESTOS DE CORTA Y \\ OPERACIONES SELVÍCOLAS} 
& 11 & Bosque claro y fuertemente aclarado. Restos de poda o aclareo ligeros (diámetro < 7.5 cm). \\
\cline{2-3}
& 12 & Predominio de los restos sobre el arbolado. La hojarasca y el matorral presente ayudarán a la propagación del fuego. \\
\cline{2-3}
& 13 & Grandes acumulaciones de restos gruesos y pesados, cubriendo todo el suelo. \\
\hline
\end{tabular}%
}
\caption{Descripción de los modelos de combustible del Inventario Forestal Nacional, clasificados por grupo funcional.}
\label{tab:modelos_combustible}
\end{table}


\subsection{Anexo: Categoría de Desarrollo y Densidad}\label{sec:CatDesDensidad}

La categoría de desarrollo se identifica en función de la altura y el diámetro de los pies de las diferentes especies. Cuando el 85\% de los ejemplares pertenecen a una determinada categoría, se considerarán todos dentro de la misma.

\begin{enumerate}
    \item Pies con altura inferior a 30 cm.
    \item Pies con altura comprendida entre 30 y 130 cm.
    \item Pies con altura superior a 130 cm y diámetro normal menor de 2,5 cm.
    \item Pies con altura superior a 130 cm y diámetro normal comprendido entre 2,5 y 7,5 cm. Corresponde a los pies menores del IFN-2.
\end{enumerate}
\vspace{0.5em}

La densidad se cuantifica según la categoría de desarrollo:

\textbf{Para las categorías 1, 2 y 3} (radio de parcela = 5 m):
\begin{enumerate}
    \item \textbf{Escasa:} De 1 a 4 pies en la parcela.
    \item \textbf{Normal:} De 5 a 15 pies en la parcela.
    \item \textbf{Abundante:} Más de 15 pies en la parcela.

\end{enumerate}

\textbf{Para la categoría 4:}
\begin{itemize}
    \item Se cuenta el número exacto de pies por especie en la subparcela de 5 m de radio. Se registra en la casilla ``N'' y se calcula aproximadamente la altura media total de cada grupo.
\end{itemize}

\vspace{0.5em}

\textit{Nota:} Si aparecen más de 40 pies en las categorías 1, 2 o 3, el conteo puede ser estimado. Los pies menores muertos no se contabilizan. Para brotes de cepa, cada uno se considera como una planta.




\subsection*{Anexo: Tipo de Regeneración}\label{sec:TipoReg}

Se identifica el origen de los pies con la siguiente clave:

\begin{enumerate}
    \item \textbf{Siembra o semilla.}
    \item \textbf{Plantación.}
    \item \textbf{Brote de cepa o raíz.}
    \item \textbf{Desconocido.}
    \item \textbf{Dudoso.}
    \item \textbf{Mixto.}
\end{enumerate}


\subsection*{Anexo: Estado de las Poblaciones (IFN3 e IFN4)}\label{sec:EstadoIFN34}

Se determinará las fases de desarrollo de las \textit{poblaciones} codificándose de la siguiente forma:

\begin{enumerate}
    \item \textbf{Repoblado}. Conjunto de pies que desde el estrato herbáceo llega hasta el subarbustivo y los pies inician la tangencia de copas.
    \item \textbf{Monte bravo}. Comprende desde el estrato y clase de edad anterior hasta el momento en que por efecto del crecimiento, los pies empiezan a perder las ramas inferiores; es decir que en esta clase de edad, las ramas se encuentran a lo largo de todo el fuste.
    \item \textbf{Latizal}. Comprende desde la clase anterior hasta que los pies tienen 20 cm de diámetro normal; es decir, el diámetro de su fuste, medido a la altura de 1,30 m del suelo.
    \item \textbf{Fustal}. Se caracteriza esta clase de edad, porque sus pies tienen diámetros normales superiores a 20 cm.
\end{enumerate}


\subsection{Anexo: Forma Principal de Masa (IFN3 e IFN4)}\label{sec:FPMasa}

\begin{enumerate}
    \item \textbf{Coetánea}. Cuando al menos el 90\% de sus pies tienen la misma edad individual. Ejemplo típico: las repoblaciones.
    \item \textbf{Regular}. Cuando al menos el 90\% de sus pies pertenecen a la misma clase artificial de edad o misma clase diamétrica en su defecto.
    \item \textbf{Semirregular}. Cuando al menos el 90\% de sus pies pertenecen a dos clases artificiales de edad cíclicamente contiguas o dos clases diamétricas contiguas en su defecto.
    \item \textbf{Irregular}. Cuando no se cumplen las condiciones anteriores, es decir, cuando en cualquier parte de la masa existen pies más o menos mezclados, de todas las clases de edad que tiene la masa o de varias clases diamétricas en su defecto.
\end{enumerate}

\subsection{Anexo: Tratamiento de la Masa (IFN3 e IFN4)}\label{sec:tratmasa}

\begin{enumerate}
    \item \textbf{Monte alto}. Cuando todos los pies proceden de semilla.
    \item \textbf{Monte medio}. Cuando coexisten pies de la misma especie, unos procedentes de semilla (brinzales) y otros de brote (chirpiales).
    \item \textbf{Monte bajo}. Cuando todos los pies proceden de brote de cepa o de raíz.
\end{enumerate}


\subsection{Anexo: Origen de la Masa (IFN3 e IFN4)}\label{sec:OrgMasa}

\begin{enumerate}
    \item \textbf{Natural}. Bosque desarrollado espontáneamente, sin intervención humana directa.
    \item \textbf{Artificial}. Plantado intencionadamente por el ser humano.
    \item \textbf{Naturalizado}. Bosque originalmente plantado pero que ha evolucionado hacia una estructura más similar a un bosque natural.
\end{enumerate}

\subsection{Anexo: MASA (IFN2)}\label{sec:masIFN2}
\begin{itemize}
    \item 1. Artificial.
    \item 2. Natural regular.
    \item 3. Natural irregular.
    \item 9. Dudoso.
\end{itemize}

\subsection{Anexo: Estado (IFN2)}\label{sec:EstadoIFN2}
\begin{itemize}
    \item 0. Repoblado. 
    \item 1. Monte bravo-repoblado. 
    \item 2. Monte bravo. 
    \item 3. Latizal-monte bravo. 
    \item 4. Latizal.
    \item 5. Fustal-latizal. 
    \item 6. Fustal.
\end{itemize}

\subsection{Anexo: Origen (IFN2)}\label{sec:OrigenIFN2}
\begin{enumerate}
    \item Siembra o semilla. 
    \item Plantación.
    \item Brote de cepa o raíz. 
    \item Desconocido.
    \item Dudoso. 
    \item Mixto.
\end{enumerate}






























\newpage
\subsection{Anexo: Código de las especies}\label{sec:especies}

\begin{landscape}
\begin{longtable}{|c|p{4cm}|p{4cm}|p{4cm}|p{4cm}|}
\caption{Relación de especies con claves IFN3 e IFN4. Esta tabla se construye cruzando información de las tablas \texttt{CambioEspecie} de la base de datos de Sig de IFN3 e IFN4 y la tabla \texttt{ESPECIES ARBÓREAS Y ARBUSTIVAS} incluida tanto en el documentador de Sig del IFN4 como en el documentador de Sig del IFN3.} \\
\hline
\textbf{Clave} & \textbf{Nombre especie} & \textbf{Sinonimias} & \textbf{Claves IFN3} & \textbf{Claves IFN4} \\
\hline
\endfirsthead
\hline \textbf{Clave} & \textbf{Nombre especie} & \textbf{Sinonimias} & \textbf{Claves IFN3} & \textbf{Claves IFN4} \\
\hline
\endhead
1 & Heberdenia bahamensis & Heberdenia excelsa &  &  \\
\hline
2 & Amelanchier ovalis & Guillomo &  &  \\
\hline
3 & Frangula alnus & Rhamnus frangula &  &  \\
\hline
4 & Rhamnus alaternus & Aladierno &  &  \\
\hline
5 & Euonymus europaeus &  &  &  \\
\hline
6 & Myrtus communis &  &  &  \\
\hline
7 & Acacia spp. &  & [7, 92, 207, 307] & [7, 90, 99, 207, 307] \\
\hline
8 & Phillyrea latifolia &  &  & [8, 90, 99, 999] \\
\hline
9 & Cornus sanguinea &  &  &  \\
\hline
10 & Sin asignar & Sin asignar &  &  \\
\hline
11 & Ailanthus altissima & Ailanthus glandulosa &  &  \\
\hline
12 & Malus sylvestris &  &  & [12, 70, 99, 999] \\
\hline
13 & Celtis australis &  &  & [13, 99] \\
\hline
14 & Taxus baccata &  &  & [14, 19, 21, 22] \\
\hline
15 & Crataegus spp. &  & [4, 15, 95, 215, 295, 315] & [15, 99, 215] \\
\hline
16 & Pyrus spp. &  &  & [16, 70, 99, 395, 999] \\
\hline
17 & Cedrus atlantica &  &  &  \\
\hline
18 & Chamaecyparis lawsoniana &  & [18, 319] & [18, 19, 28, 34, 35] \\
\hline
19 & Otras coníferas &  & [14, 17, 18, 19, 21, 22, 23, 24, 25, 26, 28, 31, 32, 33, 34, 35, 36, 37, 38, 39, 217, 219, 235, 236, 237, 238, 239, 319, 336, 337, 436, 926] &  \\
\hline
20 & Pinos &  &  & [20, 26] \\
\hline
21 & Pinus sylvestris &  & [14, 17, 19, 21, 22, 25, 31, 32, 36, 37, 237] & [19, 21, 26] \\
\hline
22 & Pinus uncinata & Pinus montana, Pinus mugo & [14, 22] & [19, 21, 22] \\
\hline
23 & Pinus pinea &  & [14, 17, 20, 23, 24, 25, 26, 36, 37, 38, 39, 219, 236, 237, 317, 319, 436] & [19, 23, 24, 26] \\
\hline
24 & Pinus halepensis &  & [17, 20, 21, 23, 24, 25, 26, 36, 37, 38, 39, 236, 237, 336] & [19, 24] \\
\hline
25 & Pinus nigra & Pinus laricio, Pinus clusiana & [14, 17, 19, 21, 25, 26, 28, 34] & [19, 21, 25, 26] \\
\hline
26 & Pinus pinaster & Pinus maritima & [14, 17, 18, 21, 23, 24, 25, 26, 27, 28, 29, 36, 37, 38, 39, 217, 219, 236, 237, 239, 317, 319, 336, 436, 826, 926] & [19, 24, 26, 28, 526, 626, 726, 826, 926] \\
\hline
27 & Pinus canariensis &  & [27] &  \\
\hline
28 & Pinus radiata & Pinus insignis & [17, 19, 21, 23, 24, 25, 26, 27, 28, 33, 34, 35, 36, 236, 337, 436] & [19, 28] \\
\hline
29 & Otros pinos &  &  &  \\
\hline
30 & Mezcla de coníferas & Coníferas, excepto pinos &  &  \\
\hline
31 & Abies alba & Abies pectinata & [31, 33] & [19, 31] \\
\hline
32 & Abies pinsapo &  &  &  \\
\hline
33 & Picea abies & Picea excelsa &  & [19, 28, 33, 34, 35] \\
\hline
34 & Pseudotsuga menziesii & Pseudotsuga douglasii & [19, 28, 31, 33, 34, 35] & [19, 28, 34, 35] \\
\hline
35 & Larix spp. &  & [33, 34, 35, 235, 335] & [19, 35] \\
\hline
36 & Cupressus sempervirens &  &  & [19, 21, 24, 26, 36] \\
\hline
37 & Juniperus communis &  &  & [21, 37, 237] \\
\hline
38 & Juniperus thurifera &  & [37, 38, 39, 237, 239] & [26, 38] \\
\hline
39 & Juniperus phoenicea &  & [36, 37, 38, 39, 237] & [26, 39, 237] \\
\hline
40 & Quercus &  & [41, 42, 43, 44, 45, 46, 48, 243] & [40, 41, 44, 999] \\
\hline
41 & Quercus robur & Quercus pedunculata & [41, 42, 43, 44, 46, 48, 243] & [41, 999] \\
\hline
42 & Quercus petraea & Quercus sessiliflora & [41, 42, 48] & [41, 42, 43, 999] \\
\hline
43 & Quercus pyrenaica & Quercus toza & [41, 42, 43, 44, 45, 46, 47, 48, 243, 746, 846, 946] & [43, 99, 999] \\
\hline
44 & Quercus faginea & Quercus lusitanica var. faginea & [40, 42, 43, 44, 46, 47, 49, 243, 946] & [43, 44, 45, 999] \\
\hline
45 & Quercus ilex ssp. ballota & Quercus rotundifolia & [2, 4, 40, 44, 45, 46, 47, 49, 66, 67, 68, 75, 91, 93, 95, 215, 269, 276, 295, 378, 476] & [41, 45, 99, 245] \\
\hline
46 & Quercus suber &  & [43, 44, 45, 46, 47, 144, 846, 946] & [43, 45, 46, 99, 646, 746, 846, 946] \\
\hline
47 & Quercus canariensis & Quercus lusitanica var. baetica & [44, 47, 49] & [47] \\
\hline
48 & Quercus rubra & Quercus borealis &  & [41, 42, 43, 48, 99, 999] \\
\hline
49 & Otros quercus &  &  & [44, 999] \\
\hline
50 & Mezcla de árboles de ribera & Árboles ripícolas & [2, 3, 4, 6, 7, 9, 51, 52, 53, 54, 55, 56, 57, 58, 59, 61, 62, 63, 69, 74, 79, 92, 94, 97, 253, 255, 256, 257, 258, 297, 299, 307, 355, 357, 392, 457, 657, 757, 857, 957] &  \\
\hline
51 & Populus alba &  & [51, 53, 55, 57, 58, 62, 255, 257, 258, 357, 657, 757, 857, 957] & [50, 51, 58, 99, 258, 999] \\
\hline
52 & Populus tremula &  &  & [50, 51, 52, 58, 99, 258, 999] \\
\hline
53 & Tamarix spp. &  &  & [50, 51, 53, 999] \\
\hline
54 & Alnus glutinosa &  & [7, 52, 53, 54] & [54, 99, 999] \\
\hline
55 & Fraxinus angustifolia &  & [55, 255] & [50, 55, 99, 255, 999] \\
\hline
56 & Ulmus minor & Ulmus campestris &  & [50, 56, 70, 99, 256, 999] \\
\hline
57 & Salix spp. &  & [3, 4, 7, 9, 51, 52, 53, 54, 55, 57, 58, 79, 92, 97, 207, 257, 258, 297, 357, 457, 557, 657, 757, 857, 957] & [57, 99, 357, 657, 999] \\
\hline
58 & Populus nigra &  &  & [50, 58, 99, 258, 999] \\
\hline
59 & Otros árboles ripícolas &  &  & [54, 59] \\
\hline
60 & Mezcla de eucaliptos & Eucaliptos & [61, 62, 63, 64] &  \\
\hline
61 & Eucalyptus globulus &  & [60, 61, 62, 63, 64, 264, 364] & [61, 99] \\
\hline
62 & Eucalyptus camaldulensis & Eucalyptus rostrata & [60, 61, 62, 63, 64, 364] & [61, 62, 64, 99, 264] \\
\hline
63 & Otros eucaliptos &  &  & [61, 63, 64, 99, 264] \\
\hline
64 & Eucalyptus nitens &  & [62, 64] & [61, 64, 99, 264] \\
\hline
65 & Ilex aquifolium &  &  & [65, 99, 999] \\
\hline
66 & Olea europaea & Olea oleaster & [66, 67] & [45, 66, 99, 999] \\
\hline
67 & Ceratonia siliqua &  &  & [45, 67] \\
\hline
68 & Arbutus unedo &  & [2, 3, 4, 5, 8, 9, 12, 13, 15, 16, 56, 65, 66, 68, 72, 76, 93, 95, 97, 99, 215, 256, 275, 276, 295, 297, 299, 315, 369, 376, 395, 399, 578] & [45, 68, 99, 999] \\
\hline
69 & Phoenix spp. &  &  &  \\
\hline
70 & Mezcla de frondosas de gran porte & Frondosas de gran porte (H.t. > 10 m) & [11, 12, 13, 16, 42, 56, 66, 71, 72, 73, 74, 75, 76, 77, 78, 256, 273, 275, 276, 277, 278, 356, 373, 376, 377, 378, 476, 478, 576, 578, 676, 678] &  \\
\hline
71 & Fagus sylvatica &  &  & [71, 99] \\
\hline
72 & Castanea sativa & Castanea vesca & [11, 16, 42, 46, 66, 72, 75, 92, 99, 299, 399, 858] & [72, 99, 999] \\
\hline
73 & Betula spp. &  & [73, 273, 373] & [73, 99, 273, 999] \\
\hline
74 & Corylus avellana &  &  & [74, 99, 999] \\
\hline
75 & Juglans regia &  &  & [70, 75, 99, 256, 999] \\
\hline
76 & Acer campestre &  & [76, 276, 376, 476, 576, 676] & [70, 76, 99, 576, 999] \\
\hline
77 & Tilia spp. &  & [77, 277, 377] & [70, 77, 377, 999] \\
\hline
78 & Sorbus spp. &  & [12, 16, 78, 278, 378, 478, 578, 678] & [50, 70, 78, 99, 278, 378, 999] \\
\hline
79 & Platanus hispanica & Platanus hybrida &  & [50, 58, 79, 99, 258, 999] \\
\hline
80 & Laurisilva &  &  &  \\
\hline
81 & Myrica faya &  &  &  \\
\hline
82 & Ilex canariensis &  & [82, 282] &  \\
\hline
83 & Erica arborea &  & [83, 283] &  \\
\hline
84 & Persea indica &  &  &  \\
\hline
85 & Sideroxylon marmulano &  &  &  \\
\hline
86 & Picconia excelsa & Notelaea excelsa &  &  \\
\hline
87 & Ocotea phoetens &  &  &  \\
\hline
88 & Apollonias barbujana & Apollonias canariensis &  &  \\
\hline
89 & Otras laurisilvas &  & [1, 86, 87, 88, 89, 95, 389, 489, 495] &  \\
\hline
90 & Mezcla de pequeñas frondosas & Frondosas de pequeño porte (H.t. $\leq$ 10 m) & [1, 2, 3, 4, 5, 6, 7, 8, 9, 11, 12, 13, 15, 16, 65, 66, 67, 68, 69, 74, 78, 91, 92, 93, 94, 95, 96, 97, 99, 215, 269, 278, 295, 297, 299, 307, 369, 378, 395, 399, 478, 495, 499, 578, 595, 599] &  \\
\hline
91 & Buxus sempervirens &  &  &  \\
\hline
92 & Robinia pseudoacacia & Acacia robinia & [7, 56, 61, 64, 72, 73, 75, 77, 79, 92, 256, 273, 373, 377] & [50, 58, 92, 99, 207, 258, 999] \\
\hline
93 & Pistacia terebinthus & Cornicabra &  &  \\
\hline
94 & Laurus nobilis & Laurel &  & [90, 94, 99] \\
\hline
95 & Prunus spp. & Prunus & [4, 5, 9, 95, 295] & [45, 90, 95, 99, 395, 999] \\
\hline
96 & Rhus coriaria & Zumaque &  & [50, 96] \\
\hline
97 & Sambucus nigra & Saúco negro &  & [50, 90, 97, 99, 657, 999] \\
\hline
98 & Carpinus betulus & Carpe & [43, 44, 46, 47, 51, 53, 54, 55, 57, 58, 60, 62, 63, 64, 72, 75, 144, 257, 357, 657] & [71, 273] \\
\hline
99 & Otras frondosas & Otras frondosas & [1, 2, 3, 4, 5, 6, 7, 8, 9, 10, 11, 12, 13, 14, 15, 16, 40, 41, 42, 43, 44, 45, 46, 47, 48, 49, 51, 52, 53, 54, 55, 56, 57, 58, 59, 60, 61, 62, 64, 65, 66, 67, 68, 69, 71, 72, 73, 74, 75, 76, 77, 78, 79, 82, 83, 84, 86, 87, 88, 89, 91, 92, 93, 94, 95, 96, 97, 99, 207, 215, 217, 243, 253, 255, 256, 257, 258, 264, 269, 273, 275, 276, 277, 278, 279, 291, 292, 293, 295, 297, 299, 307, 315, 355, 356, 357, 364, 369, 373, 375, 376, 377, 378, 389, 392, 395, 399, 415, 457, 469, 476, 478, 489, 492, 495, 499, 557, 569, 576, 578, 592, 595, 657, 676, 678, 757, 778, 857, 858, 957] & [90, 99, 999] \\
\hline
207 & Acacia melanoxylon & Acacia melanoxylon &  & [7, 99, 207, 307] \\
\hline
215 & Crataegus monogyna & Majuelo & [15, 215, 315, 415] & [15, 45, 99, 215] \\
\hline
217 & Cedrus deodara & Cedrus deodara &  & [17, 26, 217] \\
\hline
219 & Tetraclinis articulata & Tetraclinis articulata &  & [26, 219] \\
\hline
235 & Larix decidua & Alerce común &  & [19, 34, 35, 235] \\
\hline
236 & Cupressus arizonica & Ciprés arizónica &  & [19, 24, 26, 36, 236] \\
\hline
237 & Juniperus oxycedrus & Enebro oxicedro & [36, 37, 38, 39, 236, 237, 239] & [21, 26, 39, 237] \\
\hline
238 & Juniperus turbinata & Sabina canaria & [238, 337] & [38] \\
\hline
239 & Juniperus sabina & Sabina rastrera &  &  \\
\hline
243 & Quercus pubescens & Quercus pubescens, Quercus humilis & [42, 43, 44, 47, 243] & [43, 243, 999] \\
\hline
244 & Quercus lusitanica & Quercus fruticosa, Quejigueta &  &  \\
\hline
245 &  &  &  & [45, 245] \\
\hline
253 & Tamarix canariensis & Tarajal &  & [53] \\
\hline
255 & Fraxinus excelsior & Fresno excelsior & [55, 255, 355] & [50, 55, 99, 255, 999] \\
\hline
256 & Ulmus glabra & Ulmus montana &  & [70, 99, 256, 999] \\
\hline
257 & Salix alba & Sauce blanco &  & [57, 99, 257, 357, 657, 999] \\
\hline
258 & Populus x canadensis & Populus x euroamericana & [58, 258] & [50, 58, 99, 258, 999] \\
\hline
264 & Eucalyptus viminalis & Eucalipto viminalis &  & [64, 99, 264] \\
\hline
268 & Arbutus canariensis & Madroño canario &  & [68] \\
\hline
273 & Betula alba & Betula verrucosa, Abedul pubescens &  & [73, 99, 273, 999] \\
\hline
275 & Juglans nigra & Nogal &  & [256, 275] \\
\hline
276 & Acer monspessulanum & Arce de Montpelier &  & [45, 70, 276, 999] \\
\hline
277 & Tilia cordata & Tilo cordata &  & [277, 377, 999] \\
\hline
278 & Sorbus aria & Mostajo &  & [70, 99, 278, 378, 999] \\
\hline
279 & Platanus orientalis & Plátano oriental &  & [58, 999] \\
\hline
281 &  &  &  &  \\
\hline
282 & Ilex platyphylla & Naranjero &  & [82] \\
\hline
283 & Erica scoparia & Tejo, brezo arbóreo escopario &  & [83] \\
\hline
289 & Pleiomeris canariensis & Delfino &  & [89] \\
\hline
291 & Buxus balearica & Boj de Baleares &  &  \\
\hline
292 & Sophora japonica & Acacia sofora &  &  \\
\hline
293 & Pistacia atlantica & Cornicabra canaria &  & [93] \\
\hline
294 & Laurus azorica & Laurel canario & [68, 268, 294] & [94] \\
\hline
295 & Prunus spinosa & Espino negro &  &  \\
\hline
297 & Sambucus racemosa & Saúco racemosa &  &  \\
\hline
299 & Ficus carica & Higuera &  & [90, 99, 299, 999] \\
\hline
307 & Acacia dealbata & Acacia dealbata &  & [7, 90, 99, 207, 307] \\
\hline
315 & Crataegus laevigata & Espino majuelo &  & [215, 315] \\
\hline
317 & Cedrus libani & Cedrus libani &  &  \\
\hline
319 & Thuja spp. & Thuja &  & [19, 28, 319] \\
\hline
335 & Larix leptolepis & Larix kaempferi, Alerce leptolepis &  &  \\
\hline
336 & Cupressus lusitanica & Ciprés lambertiana &  & [26, 36] \\
\hline
337 & Juniperus cedrus & Enebro canario &  & [37] \\
\hline
344 &  &  &  &  \\
\hline
355 & Fraxinus ornus & Fresno orno &  &  \\
\hline
356 & Ulmus pumila & Olmo pumilo &  & [70, 356] \\
\hline
357 & Salix atrocinerea & Bardaguera &  & [57, 99, 357, 657, 999] \\
\hline
364 & Eucalyptus gomphocephalus & Eucalipto gonfo & [62, 63, 64, 264, 364] & [64, 264, 364] \\
\hline
369 & Chamaerops humilis & Palmito &  &  \\
\hline
373 & Betula pendula & Betula hispanica, Abedul péndula &  & [73, 273, 373, 999] \\
\hline
376 & Acer negundo & Negundo fraxinifolia, Arce negundo &  & [76, 99, 276, 376, 999] \\
\hline
377 & Tilia platyphyllos & Tilo común &  & [70, 377, 999] \\
\hline
378 & Sorbus aucuparia & Serbal de cazadores &  & [45, 50, 70, 78, 99, 278, 378, 999] \\
\hline
389 & Rhamnus glandulosa & Sanguino &  & [89] \\
\hline
392 & Gleditsia triacanthos & Acacia gleditsia &  &  \\
\hline
395 & Prunus avium & Cerezo silvestre &  & [90, 95, 99, 395, 999] \\
\hline
399 & Morus spp. & Morera &  &  \\
\hline
415 & Crataegus laciniata & Majoleto &  &  \\
\hline
435 & Larix x eurolepis & Alerce híbrido &  &  \\
\hline
436 & Cupressus macrocarpa & Ciprés americano &  & [26, 36, 436] \\
\hline
455 &  &  &  &  \\
\hline
456 &  &  &  & [70, 456] \\
\hline
457 & Salix babylonica & Sauce llorón &  & [57, 357, 457, 657] \\
\hline
464 &  &  &  & [61, 64, 99, 464] \\
\hline
469 & Phoenix canariensis & Palmera & [469, 569] & [69] \\
\hline
476 & Acer opalus & Arce ópalus &  & [45, 70, 276, 476, 576, 999] \\
\hline
478 & Sorbus domestica & Serbal común &  & [278, 478, 999] \\
\hline
489 & Visnea mocanera & Mocan &  & [89] \\
\hline
495 & Prunus lusitanica & Loro, hija &  & [95, 495] \\
\hline
499 & Morus alba & Morera &  &  \\
\hline
515 & Crataegus azarolus & Espino &  &  \\
\hline
557 & Salix cantabrica & Sauce cantábrico &  & [357, 557] \\
\hline
569 & Dracaena draco & Drago &  &  \\
\hline
576 & Acer pseudoplatanus & Arce seudoplátano &  & [70, 76, 99, 276, 576, 999] \\
\hline
578 & Sorbus torminalis & Serbal torminal &  & [70, 278, 378, 578, 999] \\
\hline
595 & Prunus padus & Prunus &  &  \\
\hline
599 & Morus nigra & Morera &  &  \\
\hline
657 & Salix caprea & Sauce cabruno &  & [57, 99, 357, 657, 999] \\
\hline
676 & Acer platanoides & Arce platanoide &  & [70, 99, 276, 576, 676, 999] \\
\hline
678 & Sorbus latifolia & Serbal de hoja ancha &  & [278, 678, 999] \\
\hline
757 & Salix elaeagnos & Sarga &  & [99, 357, 657, 757, 999] \\
\hline
776 &  &  &  &  \\
\hline
778 & Sorbus chamaemespilus & Serbal chame &  &  \\
\hline
857 & Salix fragilis & Mimbre &  & [357, 657, 857, 999] \\
\hline
858 & Salix canariensis & Sauce canario &  & [58] \\
\hline
917 & Cedrus spp. & Cedrus spp. &  &  \\
\hline
936 & Cupressus spp. & Cipres &  &  \\
\hline
937 & Juniperus spp. & Enebros y sabinas & [36, 37, 38, 39, 236, 237, 239, 336] &  \\
\hline
955 & Fraxinus spp. & Fresnos & [55, 255] &  \\
\hline
956 & Ulmus spp. & Olmo &  &  \\
\hline
957 & Salix purpurea & Mimbrera &  & [99, 357, 657, 957, 999] \\
\hline
975 & Juglans spp. &  &  &  \\
\hline
976 & Acer spp. & Arces & [71, 72, 76, 276, 278, 307, 378, 476, 478, 492, 576, 578, 676] &  \\
\hline
997 & Sambucus spp. &  &  &  \\
\hline
\end{longtable}
\label{tab:codificacion_especies}
\end{landscape}

