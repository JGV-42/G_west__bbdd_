\section{Conclusión}

% Resume el aporte principal del dataset.
% Reafirma su valor para la comunidad.
% Menciona brevemente la disponibilidad abierta y el potencial de reutilización.
% Puedes añadir si habrá actualizaciones o versiones futuras.

Este trabajo presenta una base de datos forestal integrada que combina información procedente del Inventario Forestal Nacional (IFN), imágenes espectrales satelitales y variables climáticas, todo ello organizado en una estructura relacional y jerárquica. La parcela forestal se establece como unidad mínima de análisis, permitiendo la vinculación precisa de distintas fuentes de datos a través de coordenadas geográficas y fechas de inventario.

\medskip

La base de datos ha sido diseñada con criterios de interoperabilidad, trazabilidad y escalabilidad, lo que facilita su explotación tanto en entornos científicos como técnicos. Su estructura modular permite acceder a diferentes niveles de agregación (por parcela, especie, clase diamétrica o estación), adaptándose a una amplia variedad de preguntas de investigación y necesidades operativas.

\medskip

A pesar de sus potenciales aplicaciones —como el modelado forestal, la evaluación de la fijación de carbono o el análisis de cambios espacio-temporales—, se reconocen algunas limitaciones. Entre ellas destacan la estimación indirecta de ciertas variables, como el carbono, y la necesidad de recurrir a modelos predictivos para completar datos faltantes. Asimismo, la representación geométrica de las parcelas se realiza a partir de supuestos circulares, en ausencia de polígonos exactos, lo que introduce una fuente de incertidumbre espacial.

\medskip

En conjunto, la base de datos ofrece una herramienta valiosa para investigadores, gestores forestales y responsables de políticas públicas interesados en la evaluación y planificación forestal basada en datos. Su desarrollo sienta las bases para futuras ampliaciones, incluyendo la incorporación de nuevas variables, la mejora de la precisión geométrica o la integración de modelos dinámicos de crecimiento forestal y cambio climático.
