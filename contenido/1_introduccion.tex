\section{Introducción}

% Introduce el contexto científico o técnico.
% Explica la importancia de los datos forestales y espectrales.
% Qué problema aborda tu base de datos.
% Qué la hace diferente o necesaria.
% A quién le puede interesar (investigadores, gestores, técnicos).

La creciente presión sobre los ecosistemas forestales provocada por el cambio climático, la intensificación de los usos del suelo y la pérdida de biodiversidad ha reforzado la necesidad de contar con datos detallados, coherentes y accesibles para la toma de decisiones y el análisis científico \cite{cmnucc1992, kioto1997}. En este contexto, disponer de bases de datos estructuradas que integren múltiples fuentes de información —desde características estructurales del bosque hasta variables espectrales y climáticas— es esencial para comprender la dinámica de los sistemas forestales y anticipar escenarios de gestión sostenible.

\medskip

El Inventario Forestal Nacional (IFN) \cite{ifn} constituye una de las principales herramientas de seguimiento de los bosques en España. Realizado periódicamente por el Ministerio para la Transición Ecológica y el Reto Demográfico (MITECO), proporciona información detallada sobre la estructura, composición y estado de la vegetación forestal. No obstante, sus datos están distribuidos en múltiples tablas y ficheros independientes, con formatos heterogéneos y estructuras no relacionales, lo que limita su interoperabilidad con otras fuentes de datos geoespaciales o climáticos.

\medskip

Al mismo tiempo, en los últimos años se ha producido una notable expansión de fuentes de datos remotos, tanto climáticos como espectrales, con acceso libre y alta resolución espacial y temporal. Algunos ejemplos clave incluyen los datos de precipitación y temperatura (\cite{copuernicus_temps, copernicus_pr}), las ortofotografías aéreas \cite{ign_pnoa}, y los índices espectrales procedentes de imágenes satelitales \cite{landsat5_data, landsat7_data, sentinel_hub_ndii, eos_indices_vegetacion}. Su integración con datos forestales de campo abre nuevas oportunidades analíticas: análisis temporales de cambio, modelización ecológica, estimación de carbono, clasificación de usos del suelo, detección de perturbaciones o entrenamiento de modelos de aprendizaje automático.

\medskip

Con el objetivo de facilitar este tipo de aplicaciones, se ha desarrollado una base de datos relacional que integra los datos del IFN con variables espectrales, climáticas, edáficas y topográficas. Esta base está organizada en torno a la parcela forestal como unidad básica, respetando una estructura jerárquica y espacio-temporal. Permite acceder a información desde un nivel general (localización y características edáficas) hasta un nivel muy específico (métricas por especie, clase diamétrica y hasta árbol, y por inventario y estación del año).

\medskip

El resultado es una base de datos modular, homogénea y optimizada para el análisis espacial y temporal. Su diseño responde tanto a necesidades técnicas (depuración, integración y escalabilidad) como a principios científicos (rigurosidad, trazabilidad, flexibilidad analítica). A diferencia del formato original del IFN, esta estructura permite realizar consultas relacionales entre inventarios, vincular datos climáticos y espectrales, y aplicar técnicas modernas de análisis de datos, incluyendo inteligencia artificial y modelización predictiva \cite{copernicus_api, miteco_abexante_2025}.


Esta herramienta puede resultar de especial interés para:
\begin{itemize}
    \item \textbf{Investigadores}, que deseen realizar análisis multivariantes, modelización ecológica o estudios longitudinales de los bosques españoles.
    \item \textbf{Técnicos forestales}, que necesiten indicadores precisos y coherentes para la planificación de actuaciones silvícolas o restauración.
    \item \textbf{Administraciones públicas y gestores}, interesados en evaluar políticas de gestión forestal, emisiones y absorciones de carbono o escenarios de cambio climático.
\end{itemize}

En este documento se detalla el proceso completo de construcción de la base de datos: desde la recopilación, transformación e integración de fuentes hasta la estructura final y sus posibles usos. También se exponen sus principales limitaciones y líneas de mejora futura.