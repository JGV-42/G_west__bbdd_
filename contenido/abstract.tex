% Resumen del artículo
% Incluye: qué contiene la base de datos, cómo se construyó, qué fuentes se usaron, y qué aporta.
% Debe ser breve (150–250 palabras) y conciso.


Este articulo presenta la construcción e integración de una base de datos forestal de carácter relacional, diseñada para facilitar el análisis ecológico, climático y territorial a partir de parcelas del Inventario Forestal Nacional (IFN) de España. La base de datos combina información de distintas fuentes y escalas, incluyendo variables geográficas, edáficas, climáticas, espectrales y estructurales del bosque, asociadas espacial y temporalmente a cada unidad de muestreo. El esquema resultante se organiza en cinco tablas principales y permite acceder a diferentes niveles de detalle: desde la parcela, hasta la especie y clase diamétrica. Se describe detalladamente el proceso de integración, las decisiones de limpieza y estructuración de datos, así como las variables generadas. También se discuten las limitaciones actuales, con especial atención a la estimación del carbono forestal, y se destacan aplicaciones potenciales en modelización forestal, análisis de cambio climático, conservación y entrenamiento de algoritmos de aprendizaje automático.

